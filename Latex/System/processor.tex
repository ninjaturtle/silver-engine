\section{processor}
varmestrømmen fra processor til kølegitterets lameller kan beskrives som en varmeledning fra processor, over kølepasta til kølegitter. 
kølepastaen vil dog her blive fraregnet, idet kølepastaens fineste formål er at maximere overfalde kontakt imellem kølesystem og processor, og undgå luftlommer, der kan lede til overophedet luft.
Aluminiums varmekonduktivitets evne indgår i kølegitteret og er på 229. Kølepastaen ses bort fra.

Tykkelserne på godset i den retning varme udbredes i er 0.5 mm for kølegitteret og regnes som en eneste massiv væg, idet processoren regnes for at have en ens temperatur i hele sin tykkelse. Varmen forsimples til at udbredes i en retning. Med disse antagelser kan varmestræmmen igennem lamellerne beregnes.
Luften antageds at flytte sig 0.01 m /s