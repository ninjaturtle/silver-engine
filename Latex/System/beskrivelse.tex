\section{Beskrivelse}

Moderne CPU og GPU enheder er bygget op af millionvis af elektroniske transistorer.  Disse transistorer er konstruret af halvledermaterialer, som typisk kun bevarer deres halvlederegenskaber indenfor et bestemt temperaturområde. Bliver disse udsat for konstante drifttemperaturer udenfor dette område, degraderer ikke bare deres elektriske signaler (med nedsat stabilitet i form af regnefejl og signalfejl) men også selve halvleder materialet de er konstrueret af, med betydelig nedsættelse af levetiden til følge.

Forfatterne noterer sig at GPU enhederne i en moderne PC fylder mere og mere i trit med at udviklingen af software og skærme betyder at der skal foretages flere og flere beregninger.  Det betyder at der rent fysisk skal bruges flere transistorer og typisk mere energi i en GPU for at opnå det ønskede ydelsesniveau.

Det betyder reelt at termodynamik i allerhøjeste grad indgår i design, funktion og brug af en computer.

I kølesystemet for en computerprocessor foregår den første varmetransmission som varmeledning fra processorens varmespreder til kølesystemet.
Ofte faciliteres denne varmetransmissionen af en kølepasta, som er med til at maksimere kontaktfladen til kølesystemet og som modvirker galvanisk tæring.
I kildelisten er inkluderet to eksempler på kølepasta, med konduktivitets værdier på hhv. 0,8 og 12,5 W/m*K.
Intentionen med kølepasta er at optimere overflade kontakten imellem kølegitter og processor og undgå at der opstår små luftfyldte områder, hvor der kan dannes ekstra varmemodstand. Dog vil laget af kølepasta her blive betragtet som liggende udenfor systemgrænsen.

Det umiddelbare indtryk er dog at GPU og CPU i termodynamisk forstand er identiske, hvorfor ordet CPU vil brugt i flæng om både CPU og GPU.

Den elektriske effekt P afsat i en CPU kan beskrives således :

 P = strømforbrug (I) * taktfrekvens (f) * kapacitans (C) .

 (kilde :ftp://download.intel.com/design/network/papers/30117401.pdf)

