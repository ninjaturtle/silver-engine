\section{konklussion}

Et varmegitter af aluminium forekommer umiddelbart at være en relativt effektiv måde at aflede varmen på. 
Den store hindring overgangsvarmen forekommer at være luft overgangsmodstanden . 
Materialet lader ikke til at gøre en synderlig forskel. 
Imidlertid er aluminium et af de mest varmeledende materialer, billigt, let og nemt at forarbejde.

Ulempen ved sådan en konstruktion her, er at filmlaget er relativt dykt og har laminart flow, hvorfor luftskiften er lidt langsom og mindre effektiv i konvektion. 
Men materialet er da underordnet ift laminart/turbolart flow og det samlede areal. 
Hvorfor man godt forstå hvorfor det bruges til pc kølesystemer. 