\section{Konklusion}

Et varmegitter af aluminium forekommer umiddelbart at være en relativt effektiv måde at aflede varmen på. 

Den store hindring i varmetransmissionenforekommer at være overgangsmodstanden for luft. 

Materialet lader ikke til at gøre en synderlig forskel, dog skal det bemærkes at aluminium er et af de mest varmeledende materialer, billigt, let og nemt at forarbejde.

Ulempen ved sådan en konstruktion her er, at fluidets filmlag er relativt tykt og har laminart flow, hvorfor luftudskiftningen er lidt langsom og mindre effektiv i konvektion. 

Men materialet er generelt underordnet ift. laminart/turbulent flow og det samlede overfladeareal, hvorfor det giver god mening at aluminium bruges i vid udstrækning til pc kølesystemer. 

Hvis antagelsen om at opdrifthastigheden på luft er $0,5mm$, så skal luften flyttes ca. 8 gange hurtigere ved at bruge en blæser, inden filmlaget bliver brudt tilstrækkeligt til at øge varmestrømmen $\phi$.