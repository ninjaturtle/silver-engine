\section{Gitter}

Fælles for alle 3 systemer er kølegitteret af aluminium, der er i kontakt med processoren. 
I simple systemer udgør den den eneste part i kølesystemet og kan selvfølgelig bestå af andre materialer end Aluminium. 
Men i rapporten her vil der bliver brugt aluminum som grundlag. 

Kølegitteret udnytter overflade areal til at bortlede varme ved konvektion fra et fast materiale(aluminium) til et et fluid(atmosfærisk luft) .
Det er forfatternes påstand at kølegitteret fungerer ved varmeledning internt i selve gitteret og varmekonvektion fra gitteret ud i den omgivende luft.
Således vil der der også være to termiske modstande, navnlig varmeledningsmodstand og varmeovergangsmodstand.

En hyppig forekommende struktur i et kølegitter er ribbe, der giver en naturlig strømning af den varme luft bort fra kølegitteret. Hvor det kun er termodynamiske kræfter der indflyldelse på strømningen.

Der præsenteres her en oversigt over de værdier der kommer i spil i til udregningerne: 

Der beregnes varmestrøm for kovektion og for stråling.

\begin{figure}
	\centering
	\includegraphics[width=0.7\linewidth]{billeder/heatsink1}
	\caption{Eksepel på kølegitter, Fra grabcad.com - bruger: Fernando}
	\label{fig:heatsink1}
\end{figure}


\begin{figure}
	\centering
	\includegraphics[width=0.7\linewidth]{billeder/lamel}
	\caption{lamel. Genstanden for hvilke, vi vil undersøge de termdynamiske forhold.Fra grabcad.com - bruger: Fernando}
	\label{fig:lamel}
\end{figure}



Tallene for ovenstående kølegitter er importeret fra en partfil i solidworks. %Kanalen for strømningen af luft der regnes for er d(1mm) bred og den hydraulisk diameter regnes som værende en snæver kanal, hvor L=2*a

H = 20 mm, $b = 25,5 mm$, $d  =1 mm$ ,Areal $A_t = 93733.73 mm^2$ , $A_{lam}= 500 mm^2$  \\ $Massen m = 31,58 gr.$ \\
varmekonduktiviten for aluminium, $\lambda_{al} = 228$ \\  
lameltykkelsen $\delta = 0.5 mm.$  \\\
Den specifikke varmekapacitet $c\_p = 1.008$ \\
stroemningshastighed lodret, i lamellets længde på $c = 0.001$ \\



For at udregne varmestrømmen $\phi$  bruges udtrykket \\ $\Phi = \alpha * A* (t\_{fl}-t\_v$ \\  også kendt som Newtons ligning. 
Hvoraf varmekonduktiviteten $\alpha$ er fundet ved tabel opslag $\alpha = ↕21.8$
Kinematisk viskositet er: $18.88$ og dynamisk viskositet = $16.92$ 
Volumenudvidelseskoefficienten $\beta_luft = 3,2$
 
Varmeledningsmodstanden : $R_l = \frac{\delta}{\lamda_{al}*A_{lam}} = 4,281$
Varmeovergangsmodstanden : $R_o = \frac{1}{\alpha_{tl}*A_{lam}} = 89.394$

Total termisk modstand er $R_{tot} = R_l + R_o = 92.225$

\section{beregninger}

Varmestrømmen $\phi$ udregnes : 

Ved at isolere $\alpha$ i ligningen Nu = $\frac{\alpha}*{\lambda}$ \\
Nu er nusselts tal og findes herunder ved at udregne Reynolds(Re) og Prandtls(pr) tal.


Strømningsforholdende for luft(fluiden) imellem lammelerne i gitteret er er fri indnvendig strømning, idet, lamellerne kan ses som kanalvægge og udgangen i en antaget lodret given retning er smal og bred. med 3 sider der åbne og derved tillader en fri udvikling af strømningen. Hvor den dydrauliske diameter er i brug og sættes lig med afstanden imellem 2 givene lameller. \approx 0.5 mm

Reference temperaturen udregnes til : 
\\$t_{film} = \frac{{t_{lam}}+t_{fluid}}{2} => t_{film}=30$
\\Grasshofs tal Gr = $\frac{g*L^3*\beta*{\deltat}}{\upsilon^2} = 8,771*10^4$
\\Prandtls tal Pl(opslag) => $Pr=0.69$


Der antages lodret  strømning af luften og Reynold tal undersøges for at få et hit om strømningsformen\\

$Re = \frac{c_{luft}*L_{hyd}}{\upsilon} = 2.648*10^{-5}$
%$ Ra = Gr*Pr = 6.052*10^4$ => \\ 
Hvilket indikerer laminar strømning over hele længden af et lamel. Hvorfor strømningen ikke undersøges yderligere, idet Re er $10^9$ mindre, end den forventede kritiske værdi for turbulent strømning.

Hastghedsprofilet hvorved den opvarmede luft stiger til vejr er udregnet til: 
$0.05*Re*L_{hd} = 0.001 $

Hvilket medfører at Nusselts tal kan udregnes som \\ 
$Nu = 3,66+ \frac{0.0668*Re*Pr*\frac{D}{L_{lam}}}{1+0.004*()\frac{D}{L_{lam}}k *Re*Pr})^{2/3} = 3.661$


Den kritisk værdi findes da ved: $Nu_m = C*Ra^4$

og $ \Phi = \alpha*A(t_{fl}-t_v =>2.597*10^6 W $\\ 
justeret til milimeter ved udtrykket: $\frac{\Phi = \alpha*A*(t_{fl}-t_v}{10^6} =>0.75 W$

samtlige lammeler afgiver da 0,75 w* 40 \approx 30 W 
Det reelle tal for hele varme gitteret anslåes til at være ca