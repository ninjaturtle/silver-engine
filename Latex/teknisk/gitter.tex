\chapter{gitter}

Fælles for alle 3 systemer er kølegitteret af aluminium, der er i kontakt med processoren. 
I simple systemer udgør den den eneste part i kølesystemet og kan selvfølgelig bestå af andre materialer end Aluminium. 
Men i rapporten her vil der bliver brugt aluminum som grundlag. 

Kølegitteret udnytter overflade areal til at bortlede varme ved konvenktion fra et fast materiale(aluminium) til et et fluid(atmosfærisk luft) .

En hyppig forekommende struktur i et køle gitter er ribbe, der giver en tvungen strømning af den varme luft bort fra kølegitteret. Som stiger til vejrs, hvor den erestattes af køligere luft. 


\include{billeder/heatsink1}


Tallene for ovenstående kølegitter er importeret fra en partfil i solidworks

Areal A=93733.73 $mm^2$  Massen m=$31,58 gr.$ og varmekonduktiviten for aluminium, $\lambda_{al}$=228  og med vægtykkelsen $\delta$ på 0.5 mm

varmestrømmen igennem gitteret kan beskrives ved : 

Sat ind i newtons ligning : $\Phi$=$\alpha*A$($t_{fl}$-$t_v$ ,  Hvor $\phi$ er i enheden $\dfrac{w}{m*K}$

$\alpha$ er varmeovergangstallet og er givet ved udtrykket $\dfrac{\lambda}{\delta}$      Så for tallene indsat i newtons ligning fås :

$\Phi$=$\dfrac{\lambda}{\delta}$*A$($t_{fl}$-$t_v$