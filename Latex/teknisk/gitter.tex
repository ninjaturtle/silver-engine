\section{gitter}

Fælles for alle 3 systemer er kølegitteret af aluminium, der er i kontakt med processoren. 
I simple systemer udgør den den eneste part i kølesystemet og kan selvfølgelig bestå af andre materialer end Aluminium. 
Men i rapporten her vil der bliver brugt aluminum som grundlag. 

Kølegitteret udnytter overflade areal til at bortlede varme ved konvektion fra et fast materiale(aluminium) til et et fluid(atmosfærisk luft) .

En hyppig forekommende struktur i et kølegitter er ribbe, der giver en naturlig strømning af den varme luft bort fra kølegitteret. Hvor det kun er termodynamiske kræfter der indflyldelse på strømningen.


\include{billeder/heatsink1}

\include{billeder/lamel}


Tallene for ovenstående kølegitter er importeret fra en partfil i solidworks. %Kanalen for strømningen af luft der regnes for er d(1mm) bred og den hydraulisk diameter regnes som værende en snæver kanal, hvor L=2*a

H =20 mm, b=25,5 mm, d=1 mm Areal A_t=93733.73 $mm^2$ , A_{lam}=500$mm$^  $Massen m=31,58 gr.$ $ varmekonduktiviten for aluminium, \lambda_{al}$=228 \\ og med vægtykkelsen $\delta$ på$ 0.5 mm$ c$_p$=1.008    $ c=0.001$

For at udregne varmestrømmen $\phi$  tages udgangspuntk i udtrykket  $\Phi$=$\alpha*A$($t_{fl}$-$t_v$ \\der også er kendt som Newtons ligning. Hvoraf $\alpha$ skal isoleres og udregnes
Kinematisk viskositet er: 18.88 og dynamisk viskositet = 16.92 
og volumenudvidelseskoefficienten $\beta=1/293$
 
 Dette gøres ved at isolere \alpha i ligningen $Nu = \frac{\alpha}*{\lambda}$
 Nu er nusselts tal og findes herunder ved at udregne Reynolds(Re) og Prandtls(pr) tal.


Strømningsforholdende for luft(fluiden) imellem lammelerne i gitteret er er fri strømning , hvorfor Grashofs og Prandtls tal udregnes.
%Til udregningerne bruges hydraulik diameter der i (termodynamik bogen) er defineret som $\frac{2*h*d}{h*d}$

Reference temperaturen udregnes til : t_{film}=\frac{{væg}+t_{fluid}}{2} => t_{film}=50
	\\Grasshofs tal Gr=$\frac{g*L^3*]$\beta$*{\deltat}}${\upsilon^2} =1.667*10^5
	\\Prandtls tal Pl(opslag) => Pr=0.083

Der antages lodret  strømning af luften og Rayleighs tal undersøges.
Ra= Gr*Pr=8.051*10^4

Den kritisk værdi findes da ved: Nu_m=C*Ra^4

og $\Phi$=$\alpha*A$($t_{fl}$-$t_v$ =>2.597*10^6 W 
$justeret til milimeter ved udtrykket:$ \frac{\Phi=\alpha*A($t_{fl}$-$t_v$}{10^6} =>2.261 W


