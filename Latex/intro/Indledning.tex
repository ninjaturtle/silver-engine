\section{Indledning}

Det termodynamiske system vi har valgt at skrive om til den skrevne projekt i termodynamik er i overordnede træk processor-kølere som bruger tør atmosfærisk luft som kølemedie.

Computere med dertilhørende køling og termodynamiske systemer fylder generelt meget i hverdagen for de fleste mennesker, men vi ved meget lidt om de termodynamiske processer i en PC.

Forfatterne interesserer sig for moderne højtydende PC-systemer, hvor køling i særdeleshed er en meget væsentlig faktor, herunder køling af den centrale processor (CPU - Central Processing Unit) samt evt. dedikerede grafikprocessorer (GPU - Graphics Processing Unit).

Vi har vi valgt at definere at systemgrænsen ligger mellem processoren og det køletekniske system, således at processoren og dens interface med det køletekniske system ligger udenfor det termodynamiske system vi beskriver.

Processoren kommer da til at definere energistrømmen ind i systemet og dermed den varme der skal transporteres ud af systemet for at det forbliver stationært ved maksimal belastning. Dette er interessant, fordi CPU og GPU enheder og den elektronik de er opbygget af typisk er designet til at fungere indenfor et bestemt temperaturområde. Bliver de udsat for konstante drifttemperaturer udenfor dette område (typisk over 60 grader celsius for CPU og over 90 grader celsius for GPU) degraderer deres transistorer - og dermed deres levetid - væsentligt, hvorfor kølesystemets kapacitet er en særdeles vigtig parameter i designet af computere.

Lauritzens og Eriksen, Termodynamik danner grundlaget for beregninger samt samt tabelopslag for diverse værdier i dette projekt.

NB: Indsæt figurer med
- Billede af moderne CPU hvor man kan se kernerne
- Billede af moderne CPU i sin normale inpakning
- Billede af en højtydende stationær computer med gennemsigtigt vindue
