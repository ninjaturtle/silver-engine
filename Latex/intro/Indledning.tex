\section{Indledning}

Computere med dertilhørende køling og termodynamiske systemer fylder generelt meget i hverdagen for de fleste mennesker, men vi ved meget lidt om de termodynamiske processer i en PC.

Forfatterne interesserer sig for moderne højtydende PC-systemer, hvor køling i særdeleshed er en meget væsentlig faktor, herunder køling af den centrale regneenhed/processor (CPU - Central Processing Unit) samt evt. dedikerede grafik regneenhed/processorer (GPU - Graphics Processing Unit).

Det termodynamiske system vi har valgt at skrive om til den skrevne projekt i termodynamik er i overordnede træk processor-kølere som bruger tør atmosfærisk luft som kølemedie.

Vi har vi valgt at definere at systemgrænsen ligger mellem processoren og det køletekniske system, således at processoren og dens interface med det køletekniske system ligger udenfor det termodynamiske system vi beskriver.

Processoren kommer da til at definere energistrømmen ind i systemet og dermed den varme der skal transporteres ud af systemet for at det forbliver stationært ved maksimal belastning. Vi vil i det følgende komme ind på hvorfor kølesystemets kapacitet ved konstant, maksimal belastning er en særdeles vigtig parameter i designet af computere.

Lauritzens og Eriksen, Termodynamik danner grundlaget for beregninger samt samt tabelopslag for diverse værdier i dette projekt.

FIXME:
Indsæt passende figurer der illustrerer vores setup.