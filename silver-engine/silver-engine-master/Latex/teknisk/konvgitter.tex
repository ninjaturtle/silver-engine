\section{beregninger}

Varmestrømmen $\phi$ udregnes : 

Ved at isolere $\alpha$ i ligningen Nu = $\frac{\alpha}*{\lambda}$ \\
Nu er nusselts tal og findes herunder ved at udregne Reynolds(Re) og Prandtls(pr) tal.


Strømningsforholdende for luft(fluiden) imellem lammelerne i gitteret er er fri indnvendig strømning, idet, lamellerne kan ses som kanalvægge og udgangen i en antaget lodret given retning er smal og bred. med 3 sider der åbne og derved tillader en fri udvikling af strømningen. 

Reference temperaturen udregnes til : 
\\$t_{film} = \frac{{væg}+t_{fluid}}{2} => t_{film}=30$
\\Grasshofs tal Gr = $\frac{g*L^3*\beta*{\deltat}}{\upsilon^2} = 8,771*10^4$
\\Prandtls tal Pl(opslag) => $Pr=0.69$


Der antages lodret  strømning af luften og Reynold tal undersøges for at få et hit om strømningsformen\\

$Re = \frac{c_{luft}*L_{hyd}}{\upsilon} = 2.648*10^{-5}$
%$ Ra = Gr*Pr = 6.052*10^4$ => \\ 
Hvilket indikerer laminar strømning over hele længden af et lamel. Hvorfor strømningen ikke undersøges yderligere, idet Re er $10^9$ mindre, end den forventede kritiske værdi for turbulent strømning.

Hastghedsprofilet hvorved den opvarmede luft stiger til vejr er udregnet til: 
$0.05*Re*L_{hd} = 0.001 $

Hvilket medfører at Nusselts tal kan udregnes som \\ 
$Nu = 3,66+ \frac{0.0668*Re*Pr*\frac{D}{L_{lam}}}{1+0.004*()\frac{D}{L_{lam}}k *Re*Pr})^{2/3} = 3.661$


Den kritisk værdi findes da ved: $Nu_m = C*Ra^4$

og $ \Phi = \alpha*A(t_{fl}-t_v =>2.597*10^6 W $\\ 
justeret til milimeter ved udtrykket: $\frac{\Phi = \alpha*A*(t_{fl}-t_v}{10^6} =>0.75 W$
